\documentclass{article}
\usepackage[utf8]{inputenc}
\usepackage[spanish]{babel}
\usepackage{listings}
\usepackage{graphicx}
\graphicspath{ {images/} }
\usepackage{cite}

\begin{document}

\begin{titlepage}
    \begin{center}
        \vspace*{1cm}
            
        \Huge
        \textbf{Parcial I}
            
        \vspace{0.5cm}
        \LARGE
        Calistenia
            
        \vspace{1.5cm}
            
        \textbf{Juan Diego Cuscagua López}
            
        \vfill
            
        \vspace{0.8cm}
            
        \Large
        Despartamento de Ingeniería Electrónica y Telecomunicaciones\\
        Universidad de Antioquia\\
        Medellín\\
        Marzo de 2021
            
    \end{center}
\end{titlepage}

\tableofcontents
\newpage
\section{Sección introductoria}\label{intro}
En la próxima sección presentaré una serie de pasos para llevar a cabo un ejercicio con el fin de poner a prueba mi capacidad para dar instrucciones y así poder ejecutar una tarea especifica

\section{Sección de Instrucciones} \label{contenido}
Para el siguiente ejercicio necesitas una hoja de papel y dos tarjetas, estas pueden ser documentos de identidad o de algún banco o entidad (preferiblemente del mismo tamaño y peso), la posición inicial de los objetos antes de iniciar el ejercicio debe ser ambas tarjetas una encima de otra bajo la hoja de papel. A partir de ahora solo puedes utilizar una mano, toma la hoja de papel y ponla sobre una superficie horizontal que sea plana, toma ambas tarjetas con tu mano, como bien sabemos, cada tarjeta es rectangular, por lo que nombraremos cada uno de sus lados para hacerlo mas facil, donde a y b son los 2 lados mas cortos de cada una, c y d son los 2 lados mas largos de cada una, ahora sosten las tarjetas de forma vertical de tal manera en que tu meñique sostenga la tarjeta a partir del lado a, el indice a partir del lado b, el dedo angular y de en medio a partir del lado c y tu pulgar en el lado d, ahora intenta separar ambos lados los cuales se encuentran sobre el meñique de tal manera en que se forme un triangulo entre ambas tarjetas de forma en que tu meñique sea la base, sin perder esta figura ubicalas sobre el centro de la hoja que pusiste anteriormente sobre una superficie horizontal plana, debes ponerlas de manera en que una tarjeta se apoye sobre la otra para que no se caigan y mantengan este triangulo donde ahora su base sería la misma hoja, si las tarjetas se caen puedes volver a intentarlo siguiendo estas instrucciones

\end{document}
